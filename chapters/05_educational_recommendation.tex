\newpage
\chapter{A Clustering-Based Method for Automatic Educational Video Recommendation Using Deep Face-Features of Actors}
\label{chap:educational_recommendation}

It is a common practice among educational content creators to make collaborative videos, that is, videos in which more than one lecturer is presenting the lecture content.
%%
Such collaborations create a network of lecturers teaching a given subject.
%%
Therefore, a method that identifies these collaborations may help students find their content of interest more easily.
%%
In this chapter, we describe the second application we investigated using the \emph{Video Face Clustering} method.
%%
We propose a recommender method based on actor's presence for educational videos. In this case, the actors are lecturers~(or teachers, professors, etc.) that are presenting an educational content on video.
%%
For instance, if a student watches a video containing lecturers A and B, our method aims at recommending other videos that contain at least one of these lecturers. 
%%
This method provides an additional aid for educational recommender systems, allowing them to use the presence of lecturers as a feature for composing their recommendations.

The remainder of this chapter is structured as follows.
Section~\ref{sec:recommendation_dataset} presents the dataset we used.
We present our method in Section~\ref{sec:recommendation_method}, followed by 
Section~\ref{sec:recommendation_experiments}, that shows the experiments to validate the face clustering and the video recommendation ranking mechanisms.
Finally, in Section~\ref{sec:recommendation_discussion}, we conclude this chapter by discussing our results.

\section{Dataset}
\label{sec:recommendation_dataset}

\section{Proposed Method}
\label{sec:recommendation_method}

\section{Experiments}
\label{sec:recommendation_experiments}

\section{Discussion}
\label{sec:recommendation_discussion}