%% -*- coding: utf-8 -*-
\newpage

\chapter{Conclusions}
\label{chap:conclusions}

In this dissertation, we present a method for spatiotemporal localization of actors called \emph{Video Face Clustering}. We investigate in what extent this method can be used as the core to leverage and enhance some innovative applications, specially in three different  practical and important tasks: Video Face Recognition, Educational Video Recommendation, and Subtitles Positioning in 360-Video.

For the Video Face Recognition, we propose a cluster-matching-based approach, derived mainly by the characteristics of our core Face Clustering method, which is very scalable since the the effort spent with annotations is significantly reduced --- as it is done over clusters instead of single images. This method uses \emph{video face clustering} and a heuristic for cluster matching in order to recognize people in video. It has achieved a recall of 99.435\% and a precision of 99.131\% when considering faces extracted from a set of 13 video files. As another consequence of face clustering, our technique can be useful for creating and labeling datasets on a less time-consuming way by labeling clusters instead of individual images.

For the Educational Video Recommendation task, we investigate a new feature that can be used for video such recommendation: the presence of specific lecturers, which again is a direct result of the application of our core method. After performing \emph{video face clustering} on each video, we extract their centroids to perform another clustering step that creates a relationship of videos that share the presence of the same lecturers. Finally, we rank the recommended videos based on the amount of time that each lecturer is present. Our method uses only the video files for performing recommendation, no other information about these videos nor the identity of the lecturers is necessary. It is worth mentioning that we do not intend to substitute other video recommendation methods, rather our application shows that, if the presence of lecturers is a relevant feature for educational video recommendation, it can be used for this purpose with a mAP value of 0.99.

For the Subtitles Positioning in 360-Video, our main contribution is the proposal of a dynamic placement of subtitles based on the automatic localization of actors and on the current visual position of the viewer. To achieve this goal, we adapted our spatiotemporal localization method to the 360 setting and created an authoring model for interactive 360-videos.