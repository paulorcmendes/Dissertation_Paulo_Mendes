%% -*- coding: utf-8; -*-

\documentclass[phd,american]{ThesisPUC}

%---------- Math ----------%
\usepackage{amsmath}
\usepackage{amssymb}
\usepackage{mathtools}
\usepackage{bbm}
%\usepackage{ulem}
%---------- Floting ----------%
\usepackage{float}
% ---------- References ----------%
%\usepackage[sort&compress,round,comma,numbers]{natbib}
%\usepackage{natbib}
\input{latex_helper}
%---------- Algorithm ----------%
\input{conf_alg}
%---------- Tables ----------%
\usepackage{multicol}
\usepackage{booktabs}
\usepackage{colortbl}
\usepackage{tabularx}
%---------- TiKz ----------%
\usepackage{threeparttable}
\usepackage{standalone}
\usepackage{xcolor}
\usepackage{tikz}
\usepackage{psfrag,epsf}
\usepackage{multirow}
\usepackage{listings}
\usepackage{soul}
%----------citation------------%
%\usepackage{apacite}

%-----------color-----------------%
\definecolor{dkblue}{rgb}{0,0,.6}
\definecolor{turquoise}{rgb}{0.25,0.87,0.82}
\definecolor{dkgreen}{rgb}{0,0.6,0}
\definecolor{dkred}{rgb}{0.7,0,0}
\definecolor{indigo}{rgb}{0.294, 0, 0.51}
\definecolor{cyan}{rgb}{0, 0.70, 0.70}
\definecolor{gray}{rgb}{0.5,0.5,0.5}
\definecolor{mauve}{rgb}{0.58,0,0.82}
\definecolor{black}{rgb}{0,0,0}

\newcommand{\pmendes}[1]{\color{red}\textbf{paulo says: }#1\color{black}}


%\pagestyle{plain}


%Para o código em EPL
\lstdefinestyle{EPLStyle}{
  numbers=left,
  numberstyle=\footnotesize\ttfamily,
  language=SQL,
  frame=tblr,
  aboveskip=0mm,
  belowskip=3mm,
  showstringspaces=false,
  columns=flexible,
  basicstyle={\small\ttfamily},
  numberstyle=\tiny\color{gray},
  keywordstyle=\color{blue},
  commentstyle=\color{gray},
  stringstyle=\color{dkgreen},
  breaklines=true,
  breakatwhitespace=true,
  tabsize=3,
  %Adiciona as keywords específicas da EPL Esper
  morekeywords={after, at, context, current_timestamp,  define, distinct, every, first, grouping, grouping_id, hour, hours, initiated, inner, instanceof, irstream, is, istream, last, match_recognize,  measures, min, minute, minutes, microsecond, microseconds, millisecond, milliseconds, msec, new, offset, output, partition, pattern, rstream, sec, second, seconds, sets, some, snapshot, sql, start, terminated, then, until, usec, using, variable, weekday, when, while, window, schema},
%
  frameround=tttt,
}

\graphicspath{{images/}}

% ---------- Cover ----------%
% Adjust the advisor's title according to gender(Prof. or Prof$^{\text{a}}$.)
\author{Paulo Renato Conceição Mendes}
\authorR{Mendes, Paulo Renato Conceição}
\advisor{Sérgio Colcher}{Prof.}
\advisorR{Colcher, Sérgio}

% This thesis will use colored figures, this goes in the catalographic sheet
\usecolour{true}
\title{Localização Espaço-temporal de Atores em Vídeos/Vídeos 360 e suas Aplicações}

\titleuk{Spatio-temporal Localization of Actors in Video/360-Video and its Applications}

\day{16$^{th}$}
\month{August}
\myyear{2021}

% CDD is the registry number of the area, given by the library. Our area (informatics) is 004.
\city{Rio de Janeiro}
\CDD{004}
\department{Informática}
\program{Informática}
\school{Centro Técnico Científico}
\university{Pontifícia Universidade Católica do Rio de Janeiro}
\uni{PUC-Rio}

%---------- Jury ----------%

% Internal jury members are declared with \jurymember{name}{title}{department}{university}
% external jury members are declared with \extjurymember{name}{title}{university}
\jury{
  \jurymember{Alberto Barbosa Raposo}{Prof.}{Departamento de Informática}{PUC-Rio}
  \extjurymember{Roberto Gerson de Albuquerque Azevedo}{Dr.}{Disney Research}
}

%---------- Front letters ----------%
\resume
{
Bachelor's degree in Computer Science at Federal University of Maranhão (UFMA) in 2019.
}

\acknowledgment
{
\noindent
Thanks to my advisor Prof. Sérgio Colcher for his guidance and support in this journey.
Thanks to my family for the endless support.
Thanks to my friends from TeleMídia Lab, for their friendship and support. To all colleagues, faculty and staff of the PUC Rio Department of Informatics for the fellowship, learning and support.
This study was financed in part by the Coordenação de Aperfeiçoamento de Pessoal de Nível Superior - Brasil (CAPES) - Finance Code 001.
}


% Workaround for keywords. The keywords in the catalographic sheet must be separated by dots, while the ones shown in the abstract must be separated by semi-colons.
% Thats why we have two commands for each language: \keywords declares the keywords for the catalographic sheet, while \keywordsabstract declares the ones for the abstract.
\keywords
{
  \key{Clusterização}
  \key{Reconhecimento Facial}
  \key{Recomendação de Vídeo}
  \key{Vídeo 360}
  
  
}

\keywordsabstract
{
  \key{Clusterização;}
  \key{Reconhecimento Facial;}
  \key{Recomendação de Vídeo;}
  \key{Vídeo 360.}
}

\keywordsuk
{

  \key{Clustering}
  \key{Face Recognition}
  \key{Video Recommendation}
  \key{360-Video}
}

\keywordsabstractuk
{
  \key{Clustering;}
  \key{Face Recognition;}
  \key{Video Recommendation;}
  \key{360-Video.}
}

\abstract{
 A popularidade de plataformas para o armazenamento e compartilhamento de vídeo tem criado um volume massivo de horas de vídeo. Dado um conjunto de atores presentes em um vídeo, a geração de metadados com a determinação temporal dos intervalos em que cada um desses atores está presente, bem como a localização no espaço 2D dos quadros em cada um desses intervalos pode facilitar a recuperação de vídeo e a recomendação. Neste trabalho, nós investigamos a \emph{Clusterização Facial em Vídeo} para a localização espaço-temporal de atores. Primeiro descrevemos nosso método de \emph{Clusterização Facial em Vídeo} em que utilizamos métodos de detecção facial, geração de \emph{embeddings} e clusterização para agrupar faces dos atores em diferentes quadros e fornecer a localização espaço-temporal destes atores. Então, nós exploramos, propomos, e investigamos aplicações inovadoras dessa localização espaço-temporal em três diferentes tarefas: (i) \emph{Reconhecimento Facial em Vídeo}, (ii) \emph{Recomendação de Vídeos Educacionais} e (iii) \emph{Posicionamento de Legendas em Vídeos 360°}. Para a tarefa (i), propomos um método baseado na similaridade de clústeres que é facilmente escalável e obteve um recall de 99.435\% e uma precisão de 99.131\% em um conjunto de vídeos. Para a tarefa (ii), propomos um método não supervisionado baseado na presença de professores em diferentes vídeos. Tal método não requer nenhuma informação adicional sobre os vídeo e obteve um valor mAP$\approx$99\%. Para a tarefa (iii), propomos o posicionamento dinâmico de legendas baseado na localização de atores em vídeo 360°.  
}

\abstractuk{
  
  The popularity of platforms for the storage and transmission of video content has created a substantial volume of video data. Given a set of actors present in a video, generating metadata with the temporal determination of the interval in which each actor is present, and their spatial 2D localization in each frame in these intervals can facilitate video retrieval and recommendation. In this work, we investigate \emph{Video Face Clustering} for this spatio-temporal localization of actors in videos. We first describe our method for \emph{Video Face Clustering} in which we take advantage of face detection, embeddings, and clustering methods to group similar faces of actors in different frames and provide the spatio-temporal localization of them. Then, we explore, propose, and investigate innovative applications of this spatio-temporal localization in three different tasks: (i) \emph{Video Face Recognition}, (ii) \emph{Educational Video Recommendation} and (iii) \emph{Subtitles Positioning in 360-video}. For (i), we propose a cluster-matching-based method that is easily scalable and achieved a recall of 99.435\% and precision of 99.131\% in a small video set. For (ii), we propose an unsupervised method based on the presence of lecturers in different videos that does not require any additional information from the videos and achieved a mAP$\approx$99\%. For (iii), we propose a dynamic placement of subtitles based on the automatic localization of actors in 360-video.
  
}


% WARNING
% The epigraph, if present, must come before the first chapter, always.
% There is a list of abreviations (abrevs.tex) which is included automatically in the ThesisPUC.cls, and is optional, comment the %% -*- coding: utf-8; -*-

\begin{thenotations}
\renewcommand{\arraystretch}{1.5}
  \noindent
  \begin{tabular}{ll}
%NCL -- Nested Context Language\\

CNN -- Convolunitonal Neural Network\\
FRPC -- Face Recognition Prize Challenge\\
HMD -- Head-Mounted Display\\
HOG -- Histogram of Oriented Gradients\\
HUD -- Heads-up Display\\
IARPA -- Intelligence Advanced Research Projects Activity\\
IJB -- IARPA Janus Benchmark-B\\
LFW -- Labeled Faces in the Wild\\
MOOC -- Massive Open Online Course\\
MTCNN -- Multitask Cascaded Convolutional Networks\\
NAN -- Neural Aggregation Network\\
VR -- Virtual Reality\\
VTE -- Virtual Teaching Environment\\

Until chap 4

  \end{tabular}

\end{thenotations} line if you do not wish to included it.
% Rationale: In the original template, the list of abreviations came before the epigraph, which caused problems with the university library, thus I've included it in the cls file.
% TODO: Declare a boolean option in the ThesisPUC.cls class in order to selectively include the abreviations list.

%%%%%%%%%%%%%%%%%%%%%%%%%%%%%%%%%%%%%%%%%%%%%%%%%%%%%%
\begin{document}

%%% -*- coding: utf-8 -*-
\newpage

\chapter{Introduction}
\label{chap:introduction}

In recent years, the popularity of platforms for the storage and transmission of video content has stimulated the production of a massive volume of video data, establishing new habits, and leveraging new applications with innovative forms of consumption of this kind of information.
%%
Just as an indicative of this huge production (and consumption) of data, we mention that, in 2019, for example, more than one billion hours of YouTube videos were watched per day.\footnote{https://kinsta.com/blog/youtube-stats/}
%%
%%Let us define relevant people present in a video file as \textit{actors}.
%%

Generating metadata with (i) the identity information of  actors present, (ii) the temporal determination of the intervals in which each of this actors are present, and (iii) their spatial localization in each of the frames along these intervals can facilitate video indexing, retrieval, recommendation and a series of other tasks which might enhance the way people interact and consume all this video data. Besides the identification, (ii) and (iii) together are what we call \textit{Spatiotemporal Localization}. 
%%
%This dissertation intends to investigate a method for this spatiotemporal localization and a series of applications that such a method enables. 

In this dissertation, we investigate a method for the spatiotemporal localization of actors in videos. Our expected contribution is two-fold: 
\begin{enumerate}
\item we propose a core process for the spatiotemporal localization in which we take advantage of face detection, embeddings and clustering methods to group similar faces (presumably from the same actors) along the frames, and \item we further explore, propose and investigate innovative application of this localization in three different practical and important tasks: Video Face Recognition, Educational Video Recommendation, and Subtitles Positioning in 360-Video.  
\end{enumerate}

The common core of this dissertation relies on its first part, in which we describe the spatiotemporal method that we call \textit{Video Face Clustering}. We describe its process composed of: \textit{frame extraction}, \textit{face detection}, \textit{embeddings generation} and \textit{clustering}. 

Based on this Video Face Clustering method for localization, we investigate three chosen applications in which we explore novel approaches, all of them enabled by that common method of localization and its benefits. 

The first application, that of face recognition, have been attracting the attention of researchers for more than two decades. Since the deep learning boom, face detection and recognition performance have greatly improved in terms of both speed and accuracy~\cite{masi2018deep}. Nowadays, face recognition systems are used in many areas such as video surveillance and security systems, video analytics systems, smart shopping, automatic face tagging in photo collections, investigative tools that search for identities in social networks based on face images, and in thousands of other applications in our daily lives.

In our work, we propose a \textit{cluster-matching-based approach} for video face recognition where clustering is used to group faces in both the face dataset and in the target video. This method was motivated (and made possible) mainly by the characteristics of our core Face Clustering method. Its main benefit is that classes (which represent different actors) do not have to be previously known, so the effort spent with annotations is significantly reduced --- as it is done over clusters instead of single images. Consequently, face recognition becomes a task of comparing clusters from the dataset with the ones extracted from images or video sources. Therefore, our approach is easily scalable and can be used to automatically generate video metadata.

The second application, that of Educational Video Recommendation, may be considered as one of a more generic class of applications referenced as \textit{recommendation system applications}, and was motivated by a shift of paradigms that we have been observing in recent years. The traditional paradigm of classroom courses, centered on the physical presence of a teacher, has been gradually giving space to online and hybrid courses, which enables the emergence of VTEs (Virtual Teaching Environment) and MOOCs~(\textit{Massive Open Online Courses}).
%%
If, on the one hand, the abundance of educational videos can contribute to and facilitate learning, on the other hand, it also makes it challenging to discover and access some specific content of interest~\cite{dias2017approach}.
This issue is usually addressed by a proactive user search (using queries, for example), or by an automatic recommendations made by specialized systems.


In general, current video recommendation methods are heavily dependent on textual information from the video, such as labels (\textit{i.e.} keywords)~\cite{mahajan2015optimising,omisore2014personalized}, or automatically generated captions \cite{barrere2020utilizaccao} from the lecturer speech. These systems face problems such as errors introduced by manually inserted labels or by imprecise speech recognition.
%%
In this work, we propose an additional feature to enhance the recommendation of educational video content which is based on actors~(in this specific case, lecturers) presence. To do that, again we take advantage of our core face clustering method. More precisely, we detect lecturers in a video taken as a reference and perform a clustering based on the face of these lecturers in different videos. Given these clusters, we extract their \textit{centroids}, and perform another clustering step for creating a relationship between videos that share the presence of the same lecturers. Finally, we rank the recommended videos based on the amount of time the referenced lecturers were present.
A particular benefit of this approach is that it can be done without supervision, allowing for new videos to be automatically analyzed.

Our third application (Subtitles Positioning in 360-Video) was motivated by the recent popularization of omnidirectional cameras and Head-Mounted-Displays (HMDs) that increased the amount of 360-video content available \cite{mendes2020authoring}. Omnidirectional videos are spherical visual signals that allow the viewer to look around a full 360-degree view of a scene from a specific point.

Several people use subtitles when consuming audiovisual media, and these subtitles are important in contributing to the understanding of the video content \cite{brown_subtitles_2017}. There are also people who choose to consume videos muted \cite{hughes_disruptive_2019}. Additionally, the work of \cite{hayati2011effect}, as referenced in \cite{hughes_disruptive_2019}, shows that consumers are more likely to watch videos entirely if they have subtitles presented with them. In traditional 2D videos, static subtitles are commonly used and they are usually placed at a fixed position, most commonly at the bottom-center of the screen \cite{rothe_dynamic_2018}.
%%
Different from traditional 2D videos, subtitles positioning in 360-videos is challenging because it involves both temporal and spatial domains \cite{agullo2019making}, and there is no fixed ``bottom-center" of the screen \cite{brown_subtitles_2017}. Most current solutions rely on positioning subtitles either statically to the viewer or at a fixed position in the 360-degree environment %\cite{mendes2020authoring}.
%%
In this work, we adapt and apply our current solution for the spatiotemporal localisation of actors to the 360-video domain. By doing that, we intend to use this localisation for positioning subtitles close to the actors in the 360-video.

\section{Structure of this Dissertation}

The remainder of this dissertation proposal is structured as follows. 
In Chapter \ref{chap:related}, we discuss some works related to the core method of our dissertation and each of the applications we investigate.
In Chapter \ref{chap:video_face_clustering}, we define our approach for spatiotemporal localization of actors, which is our core method.
The following three chapters contain the applications in which we investigate the applicability of this method.
The first application, that of \emph{Video Face Recognition}, is described in Chapter \ref{chap:face_recognition}.
In Chapter \ref{chap:educational_recommendation}, we present our application of \emph{Educational Video Recommendation}.
Our last application, that of \emph{Subtitles Positioning in 360-video}, is described in Chapter \ref{chap:subtitles_positioning}. Finally, in Chapter \ref{chap:conclusions}, we conclude this dissertation and point to possible future work.
%%% -*- coding: utf-8 -*-
\newpage

\chapter{Related Work}
\label{chap:related}

In this chapter, we make a brief review of the works related to ours. Section \ref{sec:spatiotemporal} contains work related to the core method of this work. The next three sections describe work related to the applications we propose based on our core method.

\section{Spatiotemporal Localization of Actors}
\label{sec:spatiotemporal}


The task of detecting and tracking actors in videos has been the focus of much research. In early 2004, Küblbeck and Ernst \cite{facetracking_2} used an illumination invariant approach for face detection combined with a tracking mechanism performed by means of continuous detections. Kim \emph{et al.} \cite{face_tracking} addressed the problem of tracking faces in noisy videos using a tracker that adaptively builds a target model reflecting changes in appearance, typical of a video setting. This kind of approach does not perform well in the task of spatiotemporal localization of actors because they can only track them when they are continuously present on the video. Differently, the approach we use, which is based on clustering, does not require the actors to be continuously present on the video.

More similar to ours, recent works have investigated the use of clustering for grouping faces of actors in video and, consequently, providing the spatiotemporal localization of them. Tapaswi \emph{et al.} \cite{video_face_clustering} propose Ball Cluster Learning~(BCL), a supervised approach to carve the embedding space into balls of equal size, one for each cluster. The radius of such ball is translated to a stopping criterion for iterative merging algorithms. Sharma \emph{et al.} \cite{self_supervised} propose a self-supervised Siamese network for video face clustering that can also be used in scenarios where tracks of actors are not available, such as image collections. The approach we use can also be applied to image collections, as further explained in Section \ref{chap:face_recognition}, but it differs in the sense that we use pre-trained CNNs (Convolutional Neural Networks) and traditional clustering algorithms for performing this task. 

It is worth mentioning that, in this dissertation, we do not intend to directly compare or propose a better method for the task of spatiotemporal localization of actors than the existing ones. Instead, we intend to investigate to what extent our method opens up novel approaches for the three chosen applications and the benefits that can be achieved with our approach.

\section{Video Face Recognition}
\label{sec:video_face}

Many methodologies have been proposed for \emph{Video Face Recognition}, most commonly relying on comparing selected facial features of a given image with features of faces within a database.
Using only one sample reference image of a person's face for the comparison may result in classification errors due to factors related to variations in lighting, image resolution, angle, etc.~\cite{598229}.
To overcome this problem, some face recognition approaches use multiple face samples for comparison. However, this strategy does not scale well as the complexity is a function of the number of samples.
Other approaches treat the face recognition task as a classification problem~\cite{dadi2016improved, ghosal}, where a classifier model learns rules to assign faces to previously known classes within a dataset, where each class corresponds to one person.
Nonetheless, this kind of approach does not deal well when new classes are incorporated because of the need to retrain the classification models.
Moreover, when dealing with video, these kinds of methods have to be applied to each frame, again increasing their complexity.


Traditional deep learning models for face recognition such as DeepFace~\cite{taigman2014deepface} and DeepID~\cite{sun2014deep} use a CNN with fully connected layer output to produce a representation of high-level features (face embeddings) from an input image, followed by a softmax layer to indicate the identity of classes. Other approaches, such as FaceNet~\cite{schroff2015facenet}, can directly measure the similarity among faces using euclidean space. Inspired by DeepID, this model uses the \textit{triplet loss} as the loss function to estimate similarity to one character's face to a  collection of other faces. Triplet loss improves the accuracy of the  CNN output by minimizing the euclidean distance between the anchor and the positive (face of the same identity) while maximizing the distance between the anchor and the negative (face of another identity). In this work, we evaluated different pre-trained CNN backbones on VGGFace2 dataset~\cite{cao2018vggface2} to generate the face embeddings. 

Proprietary systems for face recognition and matching are widely used by social network platforms. For instance, Facer~\cite{hazelwood2018applied} is Facebook's face detection and recognition framework. Given a photograph, it first detects all the faces. Then, it runs a  deep model to determine the likelihood of that face belonging to one of the top-N user friends. This allows  Facebook to suggest which friends the user might want to tag within the uploaded photographs. FindFace\footnote{https://findface.br.aptoide.com/app} is an app that matches photos to profile pictures on VKontakte,\footnote{https://vk.com/} a Russian social networking website similar to Facebook. FindFace uses a deep model developed by NTech Lab that won the \textit{2017 IARPA Face Recognition Prize Challenge} (FRPC)~\cite{grother20172017}  in two nominations out of three (“Identification Speed” and “Verification Accuracy”). Similarly, our method can detect faces in videos and automatically recognize their identities by a clustering-based algorithm that uses a knowledge base with the faces pre-identified as a reference; however, a comparison with such methods was not possible due to access restrictions.

Some recent works are focused on video face recognition. Pena \emph{et al.} \cite{globofacestream} proposed a face recognition system to detect characters within videos, called~\textit{Globo Face Stream}. Their method uses a Histogram of Oriented Gradients (HOG) feature combined with a linear classifier to detect faces. Next, they use  FaceNet to generate the embeddings, followed by the euclidean distance calculus to measure the similarity among faces. Yang \emph{et al.} \cite{yang2017neural} proposed a deep network for video face recognition called NAN (Neural Aggregation Network). They use a CNN to generate the embeddings, followed by an aggregation module that consists of two attention blocks that adaptively aggregate the feature vectors to form a single feature inside the convex hull spanned by them. Rao \emph{et al.} \cite{rao2017attention} proposed a method for video face recognition based on attention-aware deep reinforcement learning. They formulated the process of finding the attention of videos as a Markov decision process and training the attention model without using extra labels. Unlike existing attention models, their method takes information from both the image space and the feature space as the input to make use of face information that is discarded in the feature learning process. Sohn \emph{et al.} \cite{sohn2017unsupervised} proposed an adaptative deep learning framework for image-based face recognition and video-based face recognition. Given an embedding generated by a CNN, their framework adaptation is achieved by distilling knowledge from the network to a video adaptation network through feature matching, performing feature restoration through synthetic data augmentation, and learning a domain-invariant feature through an adversarial domain discriminator. 

Like~\cite{globofacestream, yang2017neural, rao2017attention, sohn2017unsupervised}, our method uses a CNN to generate face embeddings from face images, with the difference that it uses an unsupervised cluster-based method to compare the similarity among face datasets and faces extracted from videos. Also, our approach can detect faces that do not have an identity registered in the face dataset with excellent performance.

\section{Educational Video Recommendation}
\label{sec:recommendation}

Recommendation mechanisms are usually based on two methods: \textit{collaborative filtering} and \textit{content-based filtering}. 
In collaborative filtering, the system groups users based on their common interest in items, using users' preferences, rates, purchases, or accesses to those items. With this approach, 
knowledge about the item's content is not needed; the recommendation is purely based on the relationship between users and items.  The content-based filtering, differently, requires items' descriptions; similar items are the ones recommended to the user. Our approach fits in the latter category. In the remainder of this subsection, we describe some works devoted to the task of general \emph{Video Recommendation}. Moreover, we give an especial focus on works that share our goal of investigating \emph{Educational Video Recommendation}.

The way people watch and consume video content has been changing in the last years, moving from the traditional linear content transmission of televisions to streaming platforms. These platforms allow users to consume video content on-demand. Some examples of such platforms are YouTube,\footnote{\url{https://youtube.com}} Netflix,\footnote{\url{https://netflix.com}} Prime Video\footnote{\url{https://primevideo.com}} and Globoplay.\footnote{\url{https://globoplay.globo.com}}. In such platforms, users can retrieve video content through actively searching for their content of interest or can be presented with recommendations of such content, from which they may select one to watch. 
In this scenario, recommendations play a fundamental role in content promotion inside these platforms. It is common to use a collaborative filtering approach for recommending content to a specific user, and this kind of approach does not use any information about the content of the video. It is useful and shows good results when both the video content and user have a consumption history stored~\cite{ferreira2020investigating}. 
However, with new titles being uploaded daily to these platforms associated with their expanding user base, collaborative filtering does not perform well with these new titles and users due to the lack of consumption history~\cite{suvash14social}. 

Taking into consideration the problem of video recommendation with recently added videos, Li \emph{et al.} \cite{li2017study} propose a content-based video recommendation approach by taking advantage of CNNs to alleviate the cold-start problem. The authors represent video data with features from audio, images, and meta-data from the video content and use such content to recommend videos on a streaming platform. In \cite{lee2017large}, the authors model recommendation as a video content-based similarity learning problem, and learn deep video embeddings trained to predict video relationships identified by a co-watch-based system but using only visual and audio content. Han \emph{et al.} \cite{han2016dancelets} proposes to take advantage of the intrinsic motion information in dance videos to solve the video recommendation problem. The authors aim at recommending dance videos based on a mid-level action representation called Dancelets and use a random forest-based index to achieve fast matching of styles and to generate the final recommendation ranking of videos. Similar to \cite{li2017study} and \cite{lee2017large}, we take advantage of CNNs for extracting content from the videos and perform recommendations. Differently, our work focuses on recommending videos sharing the presence of the same actors. Similar to \cite{han2016dancelets}, our work also does not require any metadata from the video, it is solely based on the video content.

Regarding \textit{Educational Video Recommendation}, most works perform analyses and comparisons using the video textual description or speech recognition performed on them. Omisore and Samuel \cite{omisore2014personalized} propose combining \textit{fuzzy} techniques to recommend books with content suitable for students based on their reading histories in a digital library, while Mahajan \emph{et al.} \cite{mahajan2015optimising} propose, given a reference video,  mining social media, and web for suggesting links for a student to visit.
Moreover, Barrére \emph{et al.} \cite{barrere2020utilizaccao} use texts from speech recognition to create recommendations.
These works are only based on textual characteristics~(or content converted to it) for performing recommendations.
Our work focuses on using a visual part of the video, more precisely the presence of actors.


\section{Subtitles Positioning in 360-video}
\label{sec:subtitles}

We searched for works that used strategies for subtitles positioning and extracted the strategies they presented, also described as subtitling behaviour in 360-degree videos~\cite{brown_subtitles_2017}. Then, we merged the similar strategies and divided them into three main categories: \emph{screen-referenced subtitles}, \emph{world-referenced subtitles} and \emph{dynamic subtitles}. Each of these categories are described in Subsections \ref{subsec:screen_referenced}, \ref{subsec:world_referenced}, and \ref{subsection:dynamic_subtitles} respectively. Table \ref{tab:catalog} contains a summary about the strategies in each category, their advantages and disadvantages.
%%

\begingroup
%\renewcommand{\baselinestretch}{1.5}
\begin{table}[!ht]
\footnotesize
\caption{Subtitles positioning strategies catalog for 360-degree video}
\label{tab:catalog}
\hspace{-1em}
\begin{tabular}{@{}llll@{}}
\toprule
\textbf{Category}                                               & \textbf{Strategy}  & \textbf{Advantages}                                                                                    & \textbf{Disadvantages}                                                                                         \\ \midrule
\multicolumn{1}{c}{\multirow{6}{*}{\textbf{Screen-Referenced}}} & Static-Follow      & \begin{tabular}[c]{@{}l@{}}easy to locate;\\ freedom of\\ movement;\\ most common\\ strategy;\end{tabular} & issues with nausea;                                                                                            \\ \cmidrule(l){2-4} 
\multicolumn{1}{c}{}                                            & Lag-Follow         & \begin{tabular}[c]{@{}l@{}}issues with\\nausea mitigated\\ in comparison\\to static-follow;\end{tabular} & may cause rereading;                                                                                           \\ \midrule
\multirow{2}{*}{\textbf{World-Referenced}}                      & Repeated Subtitles & \begin{tabular}[c]{@{}l@{}}comfort;\\ could be\\``burnt-in'' the video;\end{tabular}                    & \begin{tabular}[c]{@{}l@{}}may cover\\important content;\\ may be confusing;\\ not always visible;\end{tabular} \\ \cmidrule(l){2-4} 
                                                                & Appear             & \begin{tabular}[c]{@{}l@{}}comfort;\\ subtitles can\\be ``dismissed";\end{tabular}                      & \begin{tabular}[c]{@{}l@{}}may be positioned in\\ spurious locations;\\ not always visible;\end{tabular}       \\ \midrule
\textbf{Dynamic}& Speaker-Following  &\begin{tabular}[c]{@{}l@{}}  help in speaker\\identification; 
\end{tabular}& not always visible;                                                                                            \\ \bottomrule
\end{tabular}
\end{table}
\endgroup

\subsection{Screen-Referenced Subtitles}
\label{subsec:screen_referenced}

In this category, the subtitles are positioned taking the screen as a reference, which can also be the viewport in an HMD. The subtitles follow the user's view and can be seen at any instant of time. We have identified two strategies following this category: \emph{static-follow} and \emph{lag-follow}. Each of these strategies is described in the following paragraphs.

\begin{figure}[!ht]
    \centering
    \includegraphics[width=0.8\textwidth]{img/video360/static-follow.png}
    \caption{Static-Follow: The sequence a, b, c demonstrates how as the user turns their head, the subtitles stay fixed to the centre of their field of view. Extracted from \cite{brown_subtitles_2017}.}.
    \label{fig:static_follow}
\end{figure}

When defining the \emph{static-follow} strategy, Brown \emph{et al.} \cite{brown_subtitles_2017} argue that it is a common behavior for showing information in Virtual Reality~(VR) experiences, as part of a ``head-up display'' (HUD). A HUD typically displays graphics that are fixed in front of the viewer at all times regardless of their posture and pose in a VR environment. Figure \ref{fig:static_follow} shows this strategy, which uses the aforementioned HUD mechanic. In this strategy, the subtitles are shown to the viewer as if they were static relative to their head, by following the viewer as they look around the environment. The subtitles are placed 15$^{\circ}$ vertically below eye level. Brown \emph{et al.} \cite{brown_subtitles_2017} mention that a possible caveat of this strategy is that some works have reported that overuse of HUD can cause issues with nausea \cite{laviola2000discussion, sharples2008virtual}.
%%
The work of Meira \emph{et al.} \cite{meira_video_2016} uses this strategy for subtitles positioning. The authors mention that the subtitles are presented at the bottom of the user's viewport and follow their gaze, but they do not mention how many degrees below eye level are used. The work of Matos \emph{et al.} \cite{matos_dynamic_2018} investigates the use of dynamic annotations in 360-degree video, with subtitles being one kind of such annotations. The authors mention the work of Brown \emph{et al.} \cite{brown_subtitles_2017} and call the \emph{static-follow} strategy by \emph{persistent}, in which subtitles~(annotations) are placed in front of the user's view. Rothe \emph{et al.} \cite{rothe_dynamic_2018} refer to this strategy as \emph{static subtitles}, and say that, in a study they conducted, this was the preferred strategy among the ones proposed by \cite{brown_subtitles_2017}. They also mention that the subtitles were positioned at 12.5$^{\circ}$ below eye level. The work of \cite{hughes_disruptive_2019} refers to this strategy as \emph{fixed position in the display picture} and mentions that it is the most common way of using subtitles in 360-degree video. Finally, Montagud \emph{et al.} \cite{montagud_culture_2020} says that there is a follow-up on the work \cite{brown_subtitles_2017} that refers to this strategy as \emph{folow head immediately}. In this follow-up work (a white paper), Brown \emph{et al.} \cite{brown2018exploring} evaluate the four strategies proposed in \cite{brown_subtitles_2017}. 

\begin{figure}[!ht]
    \centering
    \includegraphics[width=0.8\textwidth]{img/video360/lag-follow.png}
    \caption{Lag-Follow: a. Small user head movements ($<30^{\circ}$) are ignored. b. But turning beyond this boundary c. The subtitles move smoothly to the center of the field-of-view. Extracted from \cite{brown_subtitles_2017}.}
    \label{fig:lag_follow}
\end{figure}

The work of Brown \emph{et al.} \cite{brown_subtitles_2017} defines the \emph{lag-follow}~(see Figure \ref{fig:lag_follow}) strategy to address the sickness related to the \emph{static-follow} strategy while still keeping the subtitles visible to the viewer. Similar to the \emph{static-follow} strategy, the subtitles appear in front of the viewer. It remains in such position~(relative to the environment) until the viewer's head rotates more than the 30$^{\circ}$ threshold. The subtitles then smoothly rotate to be in front of the viewer again. The main objective of this strategy is to provide freedom of movement to the viewer without an immediate reaction from subtitles. However, Brown \emph{et al.} \cite{brown_subtitles_2017} say that this strategy may cause the viewer to reread the subtitles, which is not desirable.
%%
The work of Matos \emph{et al.} \cite{matos_dynamic_2018} describes a strategy that is the same as this one. It is called \emph{floating}, it starts in a position and floats into the viewer's field-of-view. Similar to what was described on the \emph{static-follow} strategy, the work of Montagud \emph{et al.} \cite{montagud_culture_2020} refers to this strategy with a different name (\emph{follow with lag}) but having the same definition.

\subsection{World-Referenced Subtitles}
\label{subsec:world_referenced}

In this category, the subtitles are positioned by taking the 360-degree environment as a reference. 
As referenced in Hughes \emph{et al.} \cite{hughes_disruptive_2019}, this category of strategy leads to better results in comfort~\cite{rothe2018positioning}. Roth \emph{et al.} \cite{rothe2018positioning}, as referenced in Hughes \emph{et al.} \cite{hughes_disruptive_2019}, also say that world-referenced subtitles conflict, in general, with the requirement that a user must always be able to read the subtitle text because it limits the user's freedom of exploring the scene.
We have identified two strategies following this category: \emph{repeated subtitles} and \emph{appear subtitles}. These strategies are described in the following paragraphs. 

\begin{figure}[!ht]
    \centering
    \includegraphics[width=0.4\textwidth]{img/video360/120_subtitles.png}
    \caption{Three repeated subtitles (having the same text) are located in the environment at 120$^{\circ}$ angles around the viewer. Extracted from \cite{brown_subtitles_2017}.}
    \label{fig:120_subtitles}
\end{figure}

In the \emph{repeated subtitles} strategy, the subtitles are placed around the user. These subtitles stay fixed in the environment and do not follow the user's head motion. Figure \ref{fig:120_subtitles} shows three repeated subtitles evenly spaced by angles of 120°. Such figure was extracted from the work of Brown \emph{et al.} \cite{brown_subtitles_2017}. The authors argue that one of the main advantages of this strategy for subtitles positioning is that the subtitles could be ``burnt-in'' to the video using a video editor. This strategy is referenced as \emph{120-degree} in the work of Brown \emph{et al.} \cite{brown_subtitles_2017}, and as \emph{evenly spaced} in the work of Montagud \emph{et al.} \cite{montagud_culture_2020}. A caveat of this strategy, mentioned by Brown \emph{et al.} \cite{brown_subtitles_2017}, is that it may cover important content located in unfortunate positions. 
%%
The work of Li \emph{et al.} \cite{li_impacts_2018} uses this strategy to evaluate the impacts of subtitles in 360-degree video journalism. They do not evaluate the subtitles positioning itself, but the impact of the subtitles. The work of Chen \emph{et al.} \cite{chen_film_2017} uses this strategy for positioning subtitles while investigating film language in society news using 360-degree videos of The New York Times. During their study, some participants found the \emph{repeated subtitles} strategy confusing as they thought, in some moments, that the subtitles in different positions had a different text.


\begin{figure}[!ht]
    \centering
    \includegraphics[width=0.8\textwidth]{img/video360/appear.png}
    \caption{Appear: a. A subtitle appears at the centre of the user's view. b. If the user moves before the subtitle is due to change, it will remain static in the environment. c. When a new subtitle is shown, it will appear at the centre of the user's view again. Extracted from \cite{brown_subtitles_2017}.}
    \label{fig:appear_subtitle}
\end{figure}

According to Brown \emph{et al.} \cite{brown_subtitles_2017}, the \emph{appear subtitles} strategy was designed based on feedback from a user who was hard of hearing, they had the idea of creating a strategy in which the viewer can dismiss the subtitle after reading it. As it is depicted in Figure \ref{fig:appear_subtitle}, the subtitles are placed at the center of the user's field of view horizontally, 15$^{\circ}$ below eye level. If the viewer moves their head, the subtitles remain static within the environment and do not follow their gaze. This strategy is also referenced as \emph{appear in front, then fixed} in the work of Montagud \emph{et al.} \cite{montagud_culture_2020}. A possible caveat of this strategy, mentioned by Brown \emph{et al.} \cite{brown_subtitles_2017}, is that the subtitles may be positioned in spurious locations if the viewer is quickly moving their head.

\subsection{Dynamic Subtitles}
\label{subsection:dynamic_subtitles}

In this category, the position of subtitles dynamically changes and depends on the scene \cite{rothe_dynamic_2018}. As referring to annotations~(that could be subtitles), Matos \emph{et al.} \cite{matos_dynamic_2018} say that there are cases where the point of interest is moving through the video, which requires a dynamic annotation that follows its movement. One strategy that we have identified in this category is the \emph{speaker-following subtitles} strategy.

In the \emph{speaker-following subtitles} strategy, the subtitles are placed close to the speaker~(see Figure \ref{fig:speaker_following}). Since the speakers may move during the video, this strategy fits in the \emph{dynamic subtitles} category. This strategy also helps in the issue of \emph{speaker identification}, as all persons in the room are visible in a 360-degree video~\cite{rothe_dynamic_2018}.  
%%
Rothe \emph{et al.} \cite{rothe_dynamic_2018} compared \emph{speaker-following} subtitles with the \emph{static-follow} strategy regarding task workload, simulator sickness and presence. For evaluating each of these dimensions, the authors used, respectively, the following questionnaires: NASA-TLX~\cite{nasa_hart1988development}; Simulator Sickness Questionaire~\cite{sickness_kennedy1993simulator}; and Presence Questionaire~\cite{presence_witmer1998measuring}. When asking which strategy the participants preferred, the authors received balanced answers. However, \emph{speaker-following} subtitles led to a higher score of presence, less sickness, and lower workload~\cite{rothe_dynamic_2018}.

\begin{figure}[!ht]
    \centering
    \includegraphics[width=0.95\textwidth]{img/video360/speaker-following.png}
    \caption{Speaker-following subtitles. Extracted from \cite{hughes_disruptive_2019}.}
    \label{fig:speaker_following}
\end{figure}

Similar to the work of Rothe \emph{et al.} \cite{rothe_dynamic_2018}, we intend to position subtitles close to the speakers in the 360-video. The main difference of our work, however, is that we automatically detect the actors present in a 360-video and use their position for placing the subtitles according to an authoring model we propose. In our authoring model, we can also determine the direction that the user is looking at. We use that to position the subtitles as \emph{static-follow} when the actor speaking is not visible to the user. In this way, the disadvantage of \emph{speaker-following} subtitles being not always visible is suppressed.
\chapter{A Method For Video Face Clustering Method}
\label{chap:video_face_clustering}

This chapter describes the core of this dissertation, which is a method for \emph{Video Face Clustering}.
It consists of detecting and grouping faces from different video frames~(ideally from the same actors) extracted from a video file.
Figure \ref{fig:video_face_clustering} depicts this process, and each of its steps is described in the next paragraphs.

\begin{figure*}[!ht]
    \centering
    \includegraphics[width=\textwidth]{img/face_clustering/video_face_clustering.pdf}
    \caption{Video face clustering process.}
    \label{fig:video_face_clustering}
\end{figure*}


First, we perform \textit{Frames Extraction} by receiving a video file as input and extracting its frames according to a given frame rate. 
These frames are used as a set of images for the next step.

The \textit{Face Detection} step uses an object detection model for detecting faces in each of its images.
In general, any object detection model can identify which, among a known set of objects, are present in the image, and provides information about their positions.
In our case, objects are faces and, therefore, the face detection model is responsible for returning the bounding boxes of the faces present in the image, specified by the $x$ and $y$ axes coordinates of the upper-left corner and of the lower-right corner of the rectangle that establishes the visual limits that encapsulate each face. 
With these bounding boxes, we can isolate and extract the bounded images, obtaining a dataset composed of images of faces only.


The objective of the \textit{Embeddings Generation} step is to represent each face image as a vector in $\mathbb{R}^{n}$.
To achieve that, it processes each of the faces generated in the previous step through a CNN, generating embeddings. 
An embedding is a representation of the input in a lower-dimensional space.
Ideally, an embedding captures some semantics of the input, e.g. by placing semantically similar inputs close together in the embedding space.
At the end of this step, we have all faces represented as embeddings.

In the \textit{Clustering} step, we group embeddings and, consequently, faces that are close in the embedding space using a clustering algorithm. 
Clustering is the task of dividing a set of data points, embeddings in this case, into groups~(clusters) such that data points in a given group are similar to other data points in the same group and dissimilar to the data points in other groups.
The clustering process should produce a partition of the dataset, i.e., each generated cluster represents a specific person, and the union of all clusters covers the whole dataset. 
It is common for clustering algorithms to require the number of clusters in advance.
However, in the context of \emph{Video Face Clustering}, we do not know this number in advance, that is the number of actors in the video. 
For that reason, we have designed a strategy for automatically choosing an adequate number of clusters.
Section \ref{subsec:unknown_nclusters} describes this strategy.

\section{Iterative Strategy for Unknown Number of Clusters}
\label{subsec:unknown_nclusters}

In this Section, we define a strategy based on the \emph{Silhouette Score} ($s$) \cite{rousseeuw1987silhouettes} to choose an adequate clustering configuration.
The \emph{Silhouette Score}~\cite{rousseeuw1987silhouettes} corresponds to the mean of the \emph{Silhouette Coefficient} of all samples.
This coefficient ($S$) for each sample is 

\begin{equation}
\label{equation:Silh}
    S = \frac{b-a}{max(a,b)} 
\vspace{1em}
\end{equation}
%%
where $a$ is the mean distance from a sample to all other samples in the same cluster, and $b$ is the mean distance from a sample to all other samples in the closest cluster to that sample.
In this way, the best value is 1 and the worst is -1. Values close to 0 indicate overlapping clusters, whereas negative values usually indicate that a sample has been assigned to the wrong cluster since a different cluster is more similar.

With this strategy~(see Algorithm 1), we increase the number of clusters until the maximum \emph{Silhouette Score} decreases no more than $t$ times in a row or until it reaches the maximum number of clusters~(lines 5-18), which is the number of embeddings~($|e|$).
The \texttt{Clustering} procedure (line 7)
can be substituted by any clustering algorithm that requires the number of clusters in advance.
When the iteration stops, we return the clustering configuration with the highest \emph{Silhouette Score}.
Since this score requires at least two clusters, it would not be possible to compute it for a clustering configuration with only one cluster~(there are only faces of a single actor).
To overcome this problem, we start with 2 clusters consecutively increasing it as described above. Then, if the returned clustering configuration has a \emph{Silhouette Score} smaller than a threshold $\omega$, that probably indicates overlapping, we say that all faces belong to one single cluster~(lines 19-20). 

\begin{algorithm}
\small
\caption{Iteratively finding the best clustering configuration for unknown number of clusters.}\label{clustering_alg}
\begin{algorithmic}[1]
\Procedure{BlindClustering}{$e, t,\omega$}\
\State $n_K\gets 1$
\State $s_{max}\gets -1$
\State $t_{cur}\gets 0$ 

\While{$t_{cur} \leq t \And n_K < |e|$}
    \State $n_K\gets n_K+1$
    \State $K_{cur}\gets Clustering(e, n_K)$
    \State $s \gets SilhouetteScore(K_{cur})$
    \If{$s < s_{max}$}
        \State $t_{cur}\gets t_{cur}+1$
    \Else
        \State $K\gets K_{cur}$
        \State $t_{cur}\gets 0$
        
        \If{$s > s_{max}$}
            \State $s_{max} \gets s$
        \EndIf
    \EndIf
\EndWhile
\If{$s_{max} < \omega$}
    \State $K\gets OneCluster(e)$
\EndIf
\State \textbf{return} $K$
\EndProcedure
\end{algorithmic}
\end{algorithm}


By the end of this process, we expect to have the spatiotemporal localization of the actors present in a video file.
Figure \ref{fig:timeline} shows an example of the results achieved by our \emph{Video Face Clustering} method. It contains a timeline of a video with lines coloring the segments that each actor appears in.

\begin{figure}[!ht]
    \centering
    \includegraphics[width=0.85\linewidth]{img/face_clustering/example_localization.png}
    \caption{Example of spatiotemporal localization.}
    \label{fig:timeline}
\end{figure}

The following three chapters describe the applications we investigate in this dissertation. 
These three applications propose novel approaches for tasks in video using spatiotemporal localization of actors through \emph{Video Face Clustering}.


\newpage
\chapter{Cluster-Matching-Based Method For Video Face Recognition}
\label{chap:face_recognition}

In this chapter, we describe the first application we investigated using the \emph{Video Face Clustering} method.
%%
We propose a cluster-matching-based approach for video face recognition where \emph{face clustering} is used to group faces in both the face dataset and in the target video~(video face clustering).
%%
Consequently, classes do not have to be previously known, and the effort spent with annotations is significantly reduced --- as it is done over clusters instead of single images.
%%
Face recognition becomes a task of comparing clusters from the dataset with the ones extracted from images or video sources.
%%
Therefore, our approach is easily scalable.

This chapter is structured as follows. 
%%
In Section \ref{sec:recognition_dataset}, we describe the dataset we produced to perform experiments. 
%%
In Section \ref{sec:recognition_method}, we detail our proposed method, followed by two sections with experiments: 
%%
Section \ref{sec:recognition_clustering_validation}, regarding the clustering methods we use, and 
%%
Section \ref{sec:recognition_matching_validation}, with the experiments with our matching heuristic.
%%
Section \ref{sec:recognition_video_evaluation} is devoted to the overall evaluation of our method and, finally, 
%%
in Section \ref{sec:recognition_discussion}, we conclude this chapter by discussing our results.

\section{Dataset}
\label{sec:recognition_dataset}
Our dataset was collected using the information provided by the Brazilian Chamber of Deputies\footnote{\url{https://www.camara.leg.br/}} for the 55th legislature, which was in session from February 1st, 2015 through January 31st, 2019.
%%
In total, 623 different deputies participated during some period in the 55th legislature, but we collected only 513 deputies from this set.

For each of the 513 deputies, we collected images in which he/she was present using Google images.
%%
The resulting dataset has a total of 9,003 images, with a mean of ~17.55 images per deputy and a standard deviation of ~6.91. 
%%
The maximum and minimum number of images per deputy are respectively 32 and 2.
%%
We have randomly selected one image of each deputy for validation and the rest for training.
%%
Thus, our dataset is divided into two: the training set, with 8,490 images, and the validation set, with 513 images.  

Besides, we created another set containing images of people who are not present in the deputies set.
%%
For that, we randomly selected 513 images from the \emph{Labeled Faces in the Wild}~(LFW)~\cite{LFWTech} dataset, that contains more than 13,000 images of faces collected from the web,
%%
and defined this subset of LFW as our \emph{Non-registered people} set.

\section{Proposed Method}
\label{sec:recognition_method}

Our method intends to recognize people in video using CNNs and clustering algorithms.
%%
Figure \ref{fig:cluster_matching} shows our proposed approach.

\begin{figure*}[!ht]
    \centering
    \includegraphics[width=\textwidth]{img/face_recognition/cluster_matching_process.pdf}
    \caption{Cluster-Matching based Method for Video Face Recognition.}
    \label{fig:cluster_matching}
\end{figure*}

In our approach we use \emph{Face Clustering} in an images dataset and in the referenced video. Then, the clusters of the images dataset are labeled. 
%%
In the \textit{Labeling} step, we assign labels~(identities) to represent the clusters.
%%
A label can be anything that represents the faces present in the cluster, e.g. a name or an id number. 
%%
Using this pipeline, instead of having to label every single face for constructing a labeled dataset, it is only necessary to label each generated cluster.
%%
Consequently, all the faces present in a cluster are assigned to the same label. 
%%
Hence, the complexity of labeling becomes a function dependent on the number of clusters, which is at most as great as the number of individuals.
%%
At the end of this step, we have a dataset of \emph{Labeled Clusters}.


Next, we perform \emph{Cluster Matching} where the clusters from the image dataset and from the video are matched using a heuristic based on clusters distance.

The \textit{Cluster Matching} step receives the set of clusters from the video and the set of labeled clusters, which is used as a reference for recognizing the clusters (and consequently the faces) in the video.
%%
%%
We designed a method based on cluster distance for performing this recognition.
%%
We compare each candidate cluster in the video with each of the labeled clusters in the reference dataset, using what we call a \emph{cluster embedding}. This \emph{cluster embedding} is computed for both the candidate cluster in the video and the reference cluster in the labeled dataset.
%%
The \emph{cluster embedding} maps each cluster to a vector in the embedding space of the face embeddings.
%%
In the experiments, we evaluate two ways of obtaining a \emph{cluster embedding}.

Let $q$ denote the \emph{cluster embedding} of the query cluster, where $q~\in~\mathbb{R}^{n}$, in which $n$ is the dimension of the embedding space.
%
Let $K$ denote the set of labeled clusters. For each $k~\in~K$, let $c_k$ denote the \emph{cluster embedding} of $k$. 
%
We compute a similarity function of $q$ and each $c_k$ for $k~\in~K$. 
%%
This similarity function is based on the Root Mean Square Error~(RMSE) function, which is largely used for computing embeddings distances in machine learning techniques.\footnote{https://www.sciencedirect.com/topics/engineering/root-mean-square-error}
%%
We decided to use the inverse of the RMSE since we want to return larger values for clusters that are closer to $q$, to be able to use these values to compute a probabilistic distribution:

\begin{equation}
\label{equation:similarity_raw}
    s_{q,k} = \frac{1}{\sqrt{\frac{1}{n}\sum_{i=0}^{n}{(q_i-c_{k,i})^2}}}
\end{equation}

Since we are not interested in clusters that too distant from $q$, we define $\overline{s}_{q,k}$ that considers only the $\alpha$ largest $s_q$ values. The $\alpha$ value is a parameter that we further evaluate in this work. 
Thus, our similarity function becomes

\begin{equation}
\label{equation:similarity}
    \overline{s}_{q,k} = \begin{cases}s_{q,k} & \text{if}~s_{q,k}~\in max_{\alpha(s_q)}\\0 & \text{otherwise}\end{cases}
\end{equation}

Given the similarity~$\overline{s}_{q,k}$, we compute the probability of $q$ being a match with each labeled cluster using the \emph{softmax} function.
%%
This function takes as input a vector of real numbers and normalizes it into a probability distribution consisting of probabilities proportional to the exponential of the input numbers. 
It is defined as

\begin{equation}
\label{equation:probability}
    p_{q,k} = \frac{e^{\overline{s}_{q,k}}}{\sum_{j~\in~K}{e^{\overline{s}_{q,j}}}}
\end{equation}

Finally, we define the function $\sigma$ that, given a \emph{cluster embedding} query $q$, returns the cluster in the labeled clusters whose \emph{cluster embedding} is more likely to have a match with the query if it has a probability greater than $0.5$. Otherwise, the query is assigned as being a match with none of the clusters ($\text{\o}$ is given as result). The $\sigma$ function is defined as follows.

\begin{equation}
\label{equation:sigma}
    \sigma{(q)} = \begin{cases}argmax(p_q) & if~~~p_{q,argmax(p_q)}~>~0.5\\\text{\o} & otherwise\end{cases}
\end{equation}

The $argmax$ function applied over $p_q$ returns the cluster $k$ whose probability of $q$ being a match with it is the greatest. 
%%
We observed that when a cluster has faces that belong to a person present in the labeled clusters, the probability tends to be higher for one single labeled cluster.
%%
However, when the person is not in any of the labeled clusters, the probability tends to be more distributed among different labeled clusters.
%%
For that reason, when none of the labeled clusters has a probability greater than $0.5$, we say that the query cluster does not match any of the labeled clusters.
%%
Hence, the person in the video represented by the query cluster is not present in the labeled clusters.

\section{Faces Clustering Validation}
\label{sec:recognition_clustering_validation}

In this section, we evaluate the quality of the clustering generated in the macro-step \emph{Face Clustering}.
%%
In the \emph{Face Detection} step, we use MTCNN~\cite{mtcnn} (Multitask
Cascaded Convolutional Networks) which is widely used for the face detection task~\cite{mtcnn1, mtcnn3}.
%%
We used the  \emph{mtcnn} Python library for its implementation\footnote{\url{https://pypi.org/project/mtcnn/}} and resized each face detected to 224x224.
%%
We tested three different CNNs for the \emph{Embeddings Generation} step in association with three different clustering algorithms for the \emph{Clustering} step.
%%
For that, we evaluated each pair of \emph{CNN x ClusteringAlg} through the quality of the clusters they generated. 
%%
To experiment, we used the training set with 8490 deputy images.

In what follows, we describe the algorithms used for the experiments, the metrics used to evaluate and the results we obtained.

\subsection{Embeddings Generation}
%\label{subsec:exp_setup}

For this step, we have three candidate CNNs that were previously trained on the VGGFace2 dataset~\cite{cao2018vggface2}. 
%%
The VGGFace2 dataset contains $3.31$ million images of $9131$ subjects and has large variations in pose, age, illumination, ethnicity, and profession.
%%
The three candidate CNNs used are VGG-16~\cite{vgg16}, ResNet-50~\cite{resnet} and SE-ResNet-50~\cite{senet}~(SeNet-50 for short). VGG-16 generates embeddings in the $\mathbb{R}^{512}$  feature space, while ResNet-50 and SeNet-50 generate embeddings in the $\mathbb{R}^{2048}$ feature space. 


\subsection{Clustering Algorithms}
%\label{subsec:clustering_algs}
%%
For this step, we selected the following clustering algorithms as candidates: K-Means~(KM)~\cite{lloyd1982least}, Affinity Propagation~(AP)~\cite{frey2007clustering}, and Agglomerative Clustering~(AC)~\cite{ward1963hierarchical}. Each of these algorithms is briefly explained next.

K-Means~\cite{lloyd1982least} is one of the most widely used unsupervised machine learning algorithms. 
%%
To process the data to be clustered, the K-Means algorithm begins with a randomly selected group of $K$ centroids, which are updated iteratively to optimize the distance of the data points to the closest centroid.
The algorithm stops when either the centroids have stabilized or the maximum number of iterations has been reached.

The Affinity Propagation algorithm~\cite{frey2007clustering}, in contrast with other clustering methods, does not require the number of clusters to be previously specified.
%%
In this algorithm, each data point sends messages to the other points informing them of their relative attractiveness to the sender. 
%%
Those targets then reply to the senders informing their availability to associate with them, considering the attractiveness of all the messages it received.
%%
The senders then reply to the targets informing the target's updated relative attractiveness.
%%
This process continues until a consensus is established.
%%
When the sender data point is associated with one of its targets, that target becomes the point's exemplar.
%%
The points with the same exemplar are assigned to the same cluster.

The Agglomerative Clustering algorithm recursively merges the pair of clusters that minimally increase a given linkage distance.
%%
The linkage criteria specify the distance to use between two sets of data points.
%%
The Agglomerative Clustering algorithm~\cite{ward1963hierarchical} merges pairs of clusters that minimize the linkage criteria.
%%
In this work, we chose the \emph{Ward} criteria~\cite{ward1963hierarchical}, which minimizes the variance of the clusters being merged.
%%
By using this method, at each step, the algorithm finds the pair of clusters that leads to a minimum increase in total within-cluster variance after merging.
%%
This increase is measured by a weighted squared distance between cluster centers.
%%
In the first step, each data point is a cluster.
%%
The clusters are merged following the criteria until the number of clusters $K$ is reached.

Different from the Affinity Propagation algorithm, K-Means and Agglomerative Clustering require the number of clusters in advance. For this case, we used 513~(number of deputies in the \emph{train set}).
%%
However, when the number of cluster is not known, a strategy for defining it is required for these two algorithms.

\subsubsection{Iterative Strategy for Unknown Number of Clusters}
\label{subsec:unknown_nclusters}

We use a strategy based on the \emph{Silhouette Score} ($s$) \cite{rousseeuw1987silhouettes} to choose the best clustering configuration.

The \emph{Silhouette Score}~\cite{rousseeuw1987silhouettes} corresponds to the mean of the \emph{Silhouette Coefficient} of all samples.
%%
This coefficient ($S$) for each sample is 
\begin{equation}
\label{equation:Silh}
    S = \frac{b-a}{max(a,b)} 
\end{equation}
%%
where $a$ is the mean distance from a sample to all other samples in the same cluster, and $b$ is the mean distance from a sample to all other samples in the closest cluster to that sample.
%%
In this way, the best value is 1 and the worst is -1. Values close to 0 indicate overlapping clusters, whereas negative values usually indicate that a sample has been assigned to the wrong cluster since a different cluster is more similar.

With this strategy~(see Algorithm 1), we increase the number of clusters until the maximum \emph{Silhouette Score} decreases no more than $t$ times in a row or until it reaches the maximum number of clusters~(lines 5-18)
, which is the number of embeddings~($|e|$).
%%
The \texttt{Clustering} procedure (line 7)
can be substituted by any clustering algorithm that requires the number of clusters in advance.
%%
When the iteration stops, we return the clustering configuration with the highest \emph{Silhouette Score}.
%%
Since this score requires at least two clusters, it would not be possible to compute it for a clustering configuration with only one cluster~(there are only faces of a single person).
%%
To overcome this problem, we start with 2 clusters consecutively increasing it as described above. Then, if the returned clustering configuration has a \emph{Silhouette Score} smaller than a threshold $\omega$, that probably indicates overlapping, we say that all faces belong to one single cluster~(lines 19-20). 

\begin{algorithm}
\small
\caption{Iteratively finding the best clustering configuration for unknown number of clusters.}\label{clustering_alg}
\begin{algorithmic}[1]
\Procedure{BlindClustering}{$e, t,\omega$}\
\State $n_K\gets 1$
\State $s_{max}\gets -1$
\State $t_{cur}\gets 0$ 

\While{$t_{cur} \leq t \And n_K < |e|$}
    \State $n_K\gets n_K+1$
    \State $K_{cur}\gets Clustering(e, n_K)$
    \State $s \gets SilhouetteScore(K_{cur})$
    \If{$s < s_{max}$}
        \State $t_{cur}\gets t_{cur}+1$
    \Else
        \State $K\gets K_{cur}$
        \State $t_{cur}\gets 0$
        
        \If{$s > s_{max}$}
            \State $s_{max} \gets s$
        \EndIf
    \EndIf
\EndWhile
\If{$s_{max} < \omega$}
    \State $K\gets OneCluster(e)$
\EndIf
\State \textbf{return} $K$
\EndProcedure
\end{algorithmic}
\end{algorithm}

\subsection{Metrics}
%\label{sec:metrics}

We evaluate the models using the V-Measure~\cite{vmeasure}, which is an entropy-based measure that computes how successfully the criteria of homogeneity and completeness have been satisfied. This metric is extensively used for comparing clustering solutions and has been used in different domain fields such as biology~\cite{bio1}, computational linguistics~\cite{nlp1}, and document engineering~\cite{doceng}.


The V-Measure is a harmonic mean of homogeneity and completeness scores, similar to how precision and recall are frequently combined into F-measure~\cite{van1979information}. It assumes a dataset comprising $N$ data points, and a set of classes, $C = \{c_i|i = 1,..., n\}$ and a set of clusters, $K = \{k_i|i = 1,...,m\}$. They also assume $A$ as a matrix produced by the clustering algorithm representing the clustering solution, such that $A = \{a_{ij}\}$ where $a_{ij}$ is the number of data points that are members of class $c_i$ and elements of cluster $k_j$.

The homogeneity is perfect when a clustering algorithm assigns only those data points that are members of a single class to a single cluster, so that the entropy is zero in each cluster.
%%

In this way, the homogeneity score determines how close a given clustering is to this ideal by examining the conditional entropy of the class distribution given the proposed clustering. 
%%
Therefore, when the clustering is perfectly homogeneous, such a value, $H(C|K) = 0$. 
 
On the other hand, when the clustering is not perfect according to this criterion, this value is proportional to the size of the dataset and the classes.
%%
For this reason, the authors normalize this value by the maximum reduction in entropy the clustering information could provide, specifically, $H(C)$.
%%
$H(C|K)$ has its maximal value when it is equal to $H(C)$ and provides no new information.
%%

Finally, to address to the convention of $1$ being desirable and $0$ undesirable, the authors define homogeneity as:

\begin{equation}
\label{equation:homo}
    h = \begin{cases} 1 & \text{if}\ H(C|K) = 0 \\1-\frac{H(C|K)}{H(C)} & \text{otherwise}\end{cases}
\end{equation}

where

\begin{equation}
    H(C|K) =  -\sum_{k=1}^{|K|}{\sum_{c=1}^{|C|}{\frac{a_{ck}}{N}log\frac{a_{ck}}{\sum_{c=1}^{|C|}{a_{ck}}}}}
\end{equation}

\vspace{1em}

\begin{equation}
    H(C) = -\sum_{c=1}^{|C|}{  \frac{\sum_{k=1}^{|K|}a_{ck}}{n}log \frac{\sum_{k=1}^{|K|}a_{ck}}{n} }
\end{equation}
\vspace{1em}

Completeness is symmetrical with respect to homogeneity, and it is perfect when a clustering assigns all data points that are members of the same class to a single cluster. 
%%
For computing such a score, the authors examine the distribution of cluster assignments within each class.
%%
Completeness is defined as 

\begin{equation}
\label{equation:completeness}
    c = \begin{cases} 1 & \text{if}\ H(K|C) = 0 \\1-\frac{H(K|C)}{H(K)} & \text{otherwise}\end{cases}
\end{equation}

where

\begin{equation}
    H(K|C) =  -\sum_{c=1}^{|C|}{\sum_{k=1}^{|K|}{\frac{a_{ck}}{N}log\frac{a_{ck}}{\sum_{k=1}^{|K|}{a_{ck}}}}}
\end{equation}
\vspace{2em}
\begin{equation}
    H(K) = -\sum_{k=1}^{|K|}{  \frac{\sum_{c=1}^{|C|}a_{ck}}{n}log \frac{\sum_{c=1}^{|C|}a_{ck}}{n} }
\end{equation}
\vspace{1em}

Finally, by using Equation \ref{equation:homo} and Equation \ref{equation:completeness}, V-measure is defined as 

\begin{equation}
    V_{\beta} = (1+\beta)\frac{h\cdot c}{\beta\cdot{h+c}}
\end{equation}
\vspace{0.7em}

The parameter $\beta$ is used to calibrate the relative importance of homogeneity and completeness when computing the V-measure. If $\beta$ is greater than 1, the completeness is more important for the V-measure, whereas 
%%
if $\beta$ is less than 1, homogeneity is more important for the V-measure. 
%%
In this work, we use $\beta = 1$, so that completeness and homogeneity contribute equally to the V-measure.

\subsection{Results}
\label{subsec:results}

Table \ref{tab:results_clustering} shows the homogeneity, completeness, and V-Measure for each combination of CNN and clustering algorithm.

From Table \ref{tab:results_clustering}, one can conclude that the best combination of CNN and clustering algorithm was SeNet-50 with Agglomerative Clustering.
%%
For this reason, we decided to use SeNet-50 for the \emph{Embeddings Generation} step and the Agglomerative Clustering algorithm for the \emph{Clustering} step.
%%
As a result, each face on the dataset is represented as an embedding in the $\mathbb{R}^{2048}$ produced by SeNet-50.

\begin{table}[!ht]
\centering
\small
\caption{Results of the evaluation of the clusters created by each combination of CNN and clustering algorithms.}
\begin{tabular}{@{}ccccc@{}}
\toprule
\textbf{CNN} & \textbf{Clustering} & \textbf{$h$} & \textbf{$c$} & \textbf{$V_1$} \\ \midrule
                  & KM                  & 0.9665                     & 0.9675                      & 0.9670             \\
ResNet-50         & AP                  & 0.0000                     & 1.0000                      & 0.0000             \\
                  & AC                  & 0.9821                     & 0.9798                      & 0.9810             \\ \midrule
                  & KM                  & 0.9725                     & 0.9726                      & 0.9725             \\
SeNet-50          & AP                  & 0.9859                     & 0.9558                      & 0.9706             \\
                  & \textbf{AC}         & \textbf{0.9862}            & \textbf{0.9833}             & \textbf{0.9847}    \\ \midrule
                  & KM                  & 0.8340                     & 0.8415                      & 0.8378             \\
VGG-16            & AP                  & 0.0000                     & 1.0000                      & 0.0000             \\
                  & AC                  & 0.8899                     & 0.8929                      & 0.8914             \\
\end{tabular}
\label{tab:results_clustering}
\end{table}

One can observe that the Affinity Propagation algorithm had a V-measure of $0$ when used with ResNet-50 and VGG-16. 
%%
This happens because, in both cases, the algorithm assigned all the data points~(face embeddings in this context) to one single cluster. 
%%
Consequently, those combinations had a homogeneity score of $0$ because there was no information gain after clustering--and completeness of 1 because data points of the same class were not scattered in different clusters.
%%
With such a combination, the V-measure, which is a harmonic mean of the homogeneity and completeness, has a value of $0$.


\section{Cluster-Matching Validation}
\label{sec:recognition_matching_validation}

Based on the results of the experiment described in Section \ref{sec:recognition_clustering_validation}, we developed another experiment in order to choose the best $\alpha$ value in Equation \ref{equation:similarity} and the best method for computing the \emph{cluster embedding} for the clusters.

As the best combination for clustering was SeNet-50 with the Agglomerative Clustering algorithm, this experiment uses the embeddings and labeled clusters generated by it.
%%
Hence, the labeled clusters are the set of faces from the 8490 deputy images with their respective embeddings and labels.
%%
Each cluster was labeled with the name of the deputy whose face is more frequent in it.

For testing, we used the \emph{Validation set}~($V$) consisting of 513 deputy images and the \emph{Non-registered people set}~($U$) also consisting of 513 images.
%%
We used these images as if they were the clusters from a video file, each cluster having a single sample.
%%
In this way, we have a more heterogeneous set for performing this experiment as if we had used a small set of video files.

We evaluate two methods for computing a \emph{cluster embedding} for the labeled clusters: \emph{cluster centroid} and \emph{sample with best silhouette coefficient}. A cluster centroid denotes the mean of the elements in a cluster.
%%
It is calculated the mean of all elements in a cluster for each axis in the embedding space. The second method uses the embedding of the sample whose silhouette coefficient is the highest.
%%
This coefficient is calculated using the mean intra-cluster distance and the mean nearest-cluster distance for each sample~(detailed in Subsection \ref{subsec:unknown_nclusters}).

\subsection{Metrics}

We evaluate the two methods for computing the \emph{cluster embedding} by assessing how well the cluster matching recognizes registered people and correctly tells when a person is not registered.
%%
A non-registered person should not have a match with any of the labeled clusters.

Let $\lambda(q)$ be the most frequent label of a cluster $q$, $|V|$ the size of the \emph{Validation set} and $|U|$ the size of the \emph{Non-registered people set}.
%%
Although $\lambda$ represents the label of a cluster in general, we also define $\Lambda$ as being the set of all the face labels present in a cluster.
%%
Let us also assume true as $1$ and false as $0$.
%%
To perform the evaluation, we compute the following four metrics:

\vspace{2em}
\noindent\begin{minipage}[c]{0.45\linewidth}
    \begin{equation}
    \small
    m_1 = \sum_{v~\in~V}{\frac{\sigma(v)\ne\text{\o}}{|V|}}
    \end{equation}
\end{minipage}
\hfill
\begin{minipage}[c]{0.45\linewidth}
    \begin{equation}
    \small
    m_2 = \sum_{v~\in~V}{\frac{\lambda(v) = \lambda(\sigma(v))}{|V|}}    
    \end{equation}
\end{minipage}

\vspace{1em}

\noindent\begin{minipage}[c]{0.45\linewidth}
    \begin{equation}
    \small
    m_3 = \sum_{v~\in~V}{\frac{\lambda(v)~\in~\Lambda(\sigma(v))}{|V|}} 
    \end{equation}
\end{minipage}
\hfill
\begin{minipage}[c]{0.45\linewidth}
    \begin{equation}
    \small
    m_4 = \sum_{u~\in~U}{\frac{\sigma(u)=\text{\o}}{|U|}}
    \end{equation}
\end{minipage}
\vspace{2em}

Where $m_1$ denotes the percentage of clusters from $V$ matched with any cluster, $m2$ the percentage of clusters from $V$ matched with a cluster with the same label, $m3$ the percentage of clusters from $V$ matched with a cluster in which the label is present, and $m4$ that denotes the percentage of clusters from $U$ that have not matched with any cluster.

\subsection{Results}

In this subsection we show the results for each of the methods for generating the \emph{cluster embeddings} in the \emph{cluster matching} step.
%%
For each of the methods, we tested different $\alpha$ values in Equation \ref{equation:similarity}.

Table \ref{tab:results_centroid} shows the results obtained using the cluster centroids as \emph{cluster embeddings}.
%%
It can be seen that with the higher the $\alpha$ value is, the lower is the percentage of registered faces assigned to the correct labeled cluster.
%%
On the other hand, the percentage of non-registered faces assigned to none of the labeled clusters increases with the $\alpha$ value.
%%
One can see that with $\alpha = 5$, we have a balance between assigning people to the correct cluster and being able to tell when a person is not in the labeled clusters.

\begin{table}[!ht]
\centering
\small
\caption{Results obtained using cluster centroids as \emph{cluster embeddings}}
\label{tab:results_centroid}
\begin{tabular}{ccccc}
\hline
\textbf{$\alpha$} & \textbf{m1} & \textbf{m2} & \textbf{m3} & \textbf{m4} \\ \hline
2 & 100.000\% & 94.932\% & 98.246\% & 0.000\% \\
3 & 97.076\% & 94.152\% & 96.881\% & 94.542\% \\
4 & 95.517\% & 93.177\% & 95.322\% & 98.635\% \\
\textbf{5} & \textbf{94.542}\% & \textbf{92.398}\% & \textbf{94.347}\% & \textbf{99.220}\% \\
6 & 93.567\% & 91.813\% & 93.372\% & 99.610\% \\
7 & 92.788\% & 91.228\% & 92.593\% & 99.805\% \\
8 & 91.813\% & 90.643\% & 91.618\% & 99.805\% \\
9 & 90.448\% & 89.279\% & 90.253\% & 99.805\% \\
10 & 89.474\% & 88.304\% & 89.279\% & 99.805\%
\end{tabular}
\end{table}

Table \ref{tab:results_silhouette} shows the results obtained using the sample with the highest silhouette coefficient as the \emph{cluster embedding} for each of the labeled clusters.
%%
One can observe that the $\alpha$ value correlates with the percentage values similar to the one observed in Table \ref{tab:results_centroid}.
%%
By using the silhouette coefficient for choosing the \emph{cluster centroids}, we have an algorithm more capable of determining when a person is not registered, achieving a percentage of $100\%$ when $\alpha>5$.
%%
However, when using this approach, the algorithm fails to correctly assign registered people to the correct labeled cluster in comparison to when it uses the cluster centroids.

\begin{table}[!ht]
\centering
\small
\caption{Results obtained using samples with the highest silhouette coefficient as \emph{cluster embeddings}}
\label{tab:results_silhouette}
\begin{tabular}{ccccc}
\hline
\textbf{$\alpha$} & \textbf{m1} & \textbf{m2} & \textbf{m3} & \textbf{m4} \\ \hline
2 & 100.000\% & 93.177\% & 94.347\% & 0.000\% \\
3 & 86.355\% & 85.575\% & 86.160\% & 98.246\% \\
4 & 76.998\% & 76.608\% & 76.998\% & 99.415\% \\
5 & 72.904\% & 72.515\% & 72.904\% & 99.805\% \\
6 & 68.811\% & 68.421\% & 68.811\% & 100.000\% \\
7 & 64.912\% & 64.522\% & 64.912\% & 100.000\% \\
8 & 61.209\% & 60.819\% & 61.209\% & 100.000\% \\
9 & 60.039\% & 59.649\% & 60.039\% & 100.000\% \\
10 & 58.285\% & 57.895\% & 58.285\% & 100.000\%
\end{tabular}
\end{table}

\section{Video Face Recognition Evaluation}
\label{sec:recognition_video_evaluation}

To evaluate our complete pipeline, we selected videos that contain only registered people (videos \emph{a} to \emph{d}), videos with both registered and non-registered people (videos \emph{e} to \emph{i}) and videos with only non-registered people (videos \emph{j} to \emph{m}).
%%
The labeled clusters are the same used in Section \ref{sec:recognition_matching_validation}, which were generated using MTCNN~\cite{mtcnn} for detecting faces, SeNet-50 for extracting its embeddings and the Agglomerative Clustering algorithm to cluster them. 
%%
The label of the clusters corresponds to the name of the deputy whose face is more frequent in it.

For performing the face identification on a video file, we first extract its frames using a frame rate of 1fps. 
%%
For each of the frames, we extract the faces present on it using MTCNN~\cite{mtcnn}. 
%%
Then, for each face identified, we resize it to 224x244, extract its embedding using SeNet-50 and cluster these faces using the Agglomerative Clustering algorithm.
%%
Since we do not know the number of people present in the video, we use the strategy described in Subsection \ref{subsec:unknown_nclusters} with the maximum time of sequential increases $t=5$, that was empirically determined to be a good stop point.

Next, we perform the \emph{Cluster Matching} with the labeled clusters and the video face clusters.
%%
We used cluster centroids as \emph{cluster embeddings} and a $\alpha=5$ as the results of Section \ref{sec:recognition_matching_validation} show that this configuration is able to correctly match the clusters while preserving the capacity of distinguishing non-registered faces.
%%
At the end of this process, each face present on the video is labeled either with the name of a registered person or as non-registered.

We evaluate our method by the Precision~(Prec), Recall (Rec), and F1-Score for the faces in the video. 
%%
As usual, the Precision is defined as the percentage of detected faces that our method correctly labels, 
%%
the Recall gives the percentage of faces that our method correctly labels among all faces in the video, and 
%%
the F1-score represents an overall performance metric based on the  harmonic mean of the precision and recall.
%%
We also count the number of exact frames match (\#EM). It corresponds to the number of frames (\#F) for which our method correctly labeled all the faces that appears. %%
Table \ref{tab:results_videos} shows the results we obtained, the number of different people in each video~(\emph{\#P}), and the number of people who are present in the labeled clusters~(\emph{\#R}).

\begin{table}[!ht]
\small
\centering
\caption{Results using the proposed approach in videos.}
\label{tab:results_videos}
\begin{tabular}{@{}cccccccc@{}}
\toprule
\textbf{Video} & \textbf{\#P} & \textbf{\#R} & \textbf{\#F} & \textbf{\#EM} & \textbf{Rec.} & \textbf{Prec.} & \textbf{F1} \\ 
\multicolumn{8}{c}{\cellcolor[HTML]{C0C0C0}{\color[HTML]{000000} videos with only registered people}}\\
a%\footnote{https://youtu.be/QjTZ\_TE1U\_g} 
& 1 & 1  & 105 & 105 & 100.000\% & 100.000\% & 100.000\%  \\
b%\footnote{https://youtu.be/4D5GGR3g\_7c}
& 1 & 1  & 80  & 79  & 100.000\% & 98.765\%  & 99.379\%   \\
c%\footnote{https://youtu.be/quYNTUsOTb8}
& 1 & 1 & 60  & 59   & 100.000\% &	98.113\% & 99.048\%   \\
d%\footnote{https://youtu.be/eB6kJYaoxHc}
& 2 & 2  & 226 & 226 & 100.000\% & 100.000\% & 100.000\%  \\ 
\multicolumn{8}{c}{\cellcolor[HTML]{C0C0C0}{\color[HTML]{000000} videos with both registered and non-registered people}}\\
e%\footnote{https://youtu.be/j07yExfJ4mA}
& 2 & 1  & 101 & 99  & 100.000\% & 98.113\%  & 99.048\%   \\
f%\footnote{https://youtu.be/Db2I1uUyDlE}
& 2 & 1  & 650 & 650 & 100.000\% & 100.000\% & 100.000\%  \\
g%\footnote{https://youtu.be/sf56sWeiMyo}
& 8 & 1  & 201 & 190 & 96.471\%  & 96.850\%  & 96.660\%   \\
h%\footnote{http://youtu.be/dYAFXogdqW4}
& 4 & 1  & 231 & 215 & 98.097\%  & 98.514\%  & 98.305\%   \\
i%\footnote{http://youtu.be/NHglWWOKmc4}
& 2 & 1  & 88  & 88  & 100.000\% & 100.000\% & 100.000\%  \\ 
\multicolumn{8}{c}{\cellcolor[HTML]{C0C0C0}{\color[HTML]{000000} videos with only non-registered people}}\\
j%\footnote{https://youtu.be/UH0nTHb6OdY}
& 6 & 0  & 625 & 617 & 99.551\%  & 99.551\%  & 99.551\%   \\
k%\footnote{http://youtu.be/wHN5vYlJ-Vk}
& 2 & 0  & 231 & 228 & 100.000\% & 98.872\%  & 99.433\%   \\
l%\footnote{http://youtu.be/WwRdjf4eEgk}
& 2 & 0  & 131 & 121 & 99.482\%  & 95.522\%  & 97.462\%   \\
m%\footnote{http://youtu.be/3dIVdsiPDH8}
& 1 & 0  & 225 & 222 & 99.556\%  & 99.115\%  & 99.335\%   \\
\bottomrule
\end{tabular}
\end{table}


One can observe that our method is able to correctly classify the faces of people that are present in the labeled clusters while also being able to tell when a person is not registered in the labeled clusters.
%%
However, Table \ref{tab:results_videos} shows that the precision was smaller for some videos. Analyzing these videos, we observed that this result was mostly due to some non-face objects that were detected as if they were faces.
%%
In \emph{Video l}, for instance, in some frames, a mug was detected as if it was a face~(Figure \ref{fig:precision}). 
%%
On the other hand, in \emph{Video g}, that had the lowest recall, one face was not detected in some frames because it was partially covered by a text~(Figure \ref{fig:recall}).

\begin{figure}[!ht]
\centering
    \begin{subfigure}{0.47\linewidth}
        \centering
        \includegraphics[width=0.9\textwidth]{img/face_recognition/maia2.png}
        \caption{Example where a face is not detected because it is covered}
        \label{fig:recall}
    \end{subfigure}\hfill
    \begin{subfigure}{0.47\linewidth}
        \centering
        \includegraphics[width=0.9\textwidth]{img/face_recognition/wagner2.png}
        \caption{Example where a mug is detected as if it was a face.}
        \label{fig:precision}
    \end{subfigure}
\caption{Cases where our approach was incorrect.}
\end{figure}

Similar to the work of Pena \emph{et. al}~\cite{globofacestream}, our method can be used to generate metadata in video files indicating the people that appear in it.
%%
Figure \ref{fig:timeline_pol} shows the first 12 seconds of \emph{Video d}~(detailed in Table \ref{tab:results_videos}) where two identified Brazilian politicians are shown in each frame with its respective colored cluster.
\begin{figure}[!ht]
    \centering
    \includegraphics[width=0.8\linewidth]{img/face_recognition/timeline_pol.png}
    \caption{Timeline with tagged frames by their clusters of registered people}
    \label{fig:timeline_pol}
\end{figure}

Besides being able to recognize people in video files, by using face embeddings and clustering, we can detect the frames where the same person appears without even knowing who the person is or if he/she is in the labeled clusters.
%%
Figure \ref{fig:timeline_non_reg} shows the first 24 seconds of \emph{Video j}~(detailed in Table \ref{tab:results_videos}) with the frames tagged with the clusters identified in each frame, where each color represents a cluster.

\begin{figure}[!ht]
    \centering
    \includegraphics[width=0.8\linewidth]{img/face_recognition/timeline2.png}
    \caption{Timeline with tagged frames by their clusters of non-registered people}
    \label{fig:timeline_non_reg}
\end{figure}


\section{Discussion}
\label{sec:recognition_discussion}

This chapter proposed a cluster-matching-based method for video face recognition. 
%%
This method uses face embeddings, clustering algorithms, and a heuristic for cluster matching in order to recognize people in video. 
%%
Our method has achieved a recall of 99.435\% and a precision of 99.131\% when considering all faces present in the frames extracted from a set of 13 video files.  
%%
Moreover, our method is also capable of determining the video segments where each person is present.
%%
With this information, it is capable of generating metadata regarding the temporal presence of actors in a video.

As a consequence of face clustering, our work also demonstrates that this technique can be useful for labeling datasets in a less time-consuming way, where clusters are labeled instead of individual data points.
%%
Although this work focuses in faces, the method proposed can be applied to other domain fields.
%%
The only requirement is that the designer has to  conceive a technique in which objects of the same class are placed closed to each other in the corresponding vector space.

One of the limitations of this work is related to the size of the video set used for the overall evaluation of our method. 
%%
This is due to the difficulty of finding videos where it is possible to manually identify in each frame whether each person present is registered or not in our labeled clusters.
%%
Other limitation is that the proposed method was not able to detect partially covered faces. 
%%
This is due to a limitation of the MTCNN~\cite{mtcnn}, which is currently not robust enough to deal with this condition.
%%
In future work, we intend to address these problems.
\newpage
\chapter{A Clustering-Based Method for Automatic Educational Video Recommendation Using Deep Face-Features of Actors}
\label{chap:educational_recommendation}

It is a common practice among educational content creators to make collaborative videos, that is, videos in which more than one lecturer is presenting the lecture content.
%%
Such collaborations create a network of lecturers teaching a given subject.
%%
Therefore, a method that identifies these collaborations may help students find their content of interest more easily.
%%
In this chapter, we describe the second application we investigated using the \emph{Video Face Clustering} method.
%%
We propose a recommender method based on actor's presence for educational videos. In this case, the actors are lecturers~(or teachers, professors, etc.) that are presenting an educational content on video.
%%
For instance, if a student watches a video containing lecturers A and B, our method aims at recommending other videos that contain at least one of these lecturers. 
%%
This method provides an additional aid for educational recommender systems, allowing them to use the presence of lecturers as a feature for composing their recommendations.

The remainder of this chapter is structured as follows.
Section~\ref{sec:recommendation_dataset} presents the dataset we used.
We present our method in Section~\ref{sec:recommendation_method}, followed by 
Section~\ref{sec:recommendation_experiments}, that shows the experiments to validate the face clustering and the video recommendation ranking mechanisms.
Finally, in Section~\ref{sec:recommendation_discussion}, we conclude this chapter by discussing our results.

\section{Dataset}
\label{sec:recommendation_dataset}

\section{Proposed Method}
\label{sec:recommendation_method}

\section{Experiments}
\label{sec:recommendation_experiments}

\section{Discussion}
\label{sec:recommendation_discussion}
\newpage
\chapter{Automatic Subtitles Positioning in 360-Video}
\label{chap:subtitles_positioning}

In \cite{mendes2020authoring}, we proposed an authoring model for interactive 360-video. In such a model, we can define interactive 360-videos that are presented together with additional information attached to it, such as image, text, 2D traditional videos and spatial audio. The positioning of such information is defined by their polar coordinates, start time and duration. For instance, we can define that a text moves with the user's head motion and is always visible or that such text is placed at a fixed position if its in user's the field of view. In this chapter we describe how \emph{video face clustering} can be used together with this authoring model for automatic subtitles positioning in 360-video. In Section \ref{sec:authoring_model}, we describe the authoring model we proposed. Section \ref{sec:authoring_clustering_360} describes how we adapt \emph{video face clustering} for the context of 360-videos. Finally, Section \ref{sec:authoring_discussion} describes how we use both the authoring model and \emph{video face clustering} for automatic positioning subtitles in 360-videos.

\section{An Authoring Model for Interactive 360-Videos}
\label{sec:authoring_model}

\section{Video Face Clustering in 360-Videos}
\label{sec:authoring_clustering_360}

\section{Discussion}
\label{sec:authoring_discussion}
%% -*- coding: utf-8 -*-
\newpage

\chapter{Conclusions}
\label{chap:conclusions}

In this dissertation, we present a method for spatiotemporal localization of actors called \emph{Video Face Clustering}. As part of this method, we also define an algorithm for finding an adequate number of clusters based on the silhouette score. We investigate in what extent this method can be used as the core to leverage and enhance some innovative applications, specially in three different  practical and important tasks: Video Face Recognition, Educational Video Recommendation, and Subtitles Positioning in 360-Video.

For the Video Face Recognition, we propose a cluster-matching-based approach, derived mainly by the characteristics of our core Face Clustering method, which is very scalable since the the effort spent with annotations is significantly reduced --- as it is done over clusters instead of single images. This method uses \emph{video face clustering} and a heuristic for cluster matching in order to recognize people in video. It has achieved a recall of 99.435\% and a precision of 99.131\% when considering faces extracted from a set of 13 video files. As another consequence of face clustering, our technique can be useful for creating and labeling datasets on a less time-consuming way by labeling clusters instead of individual images.

For the Educational Video Recommendation task, we investigate a new feature that can be used for video such recommendation: the presence of specific lecturers, which again is a direct result of the application of our core method. After performing \emph{video face clustering} on each video, we extract their centroids to perform another clustering step that creates a relationship of videos that share the presence of the same lecturers. Finally, we rank the recommended videos based on the amount of time that each lecturer is present. Our method uses only the video files for performing recommendation, no other information about these videos nor the identity of the lecturers is necessary. It is worth mentioning that we do not intend to substitute other video recommendation methods, rather our application shows that, if the presence of lecturers is a relevant feature for educational video recommendation, it can be used for this purpose with a mAP value of 0.99.

For the Subtitles Positioning in 360-Video, our main contribution is the proposal of a dynamic placement of subtitles based on the automatic localization of actors and on the current visual position of the viewer. To achieve this goal, we adapted our spatiotemporal localization method to the 360 setting and created an authoring model for interactive 360-videos. Because of the severe distortions present in equirectangular 360-videos, we used an approach based on viewports extraction for the \emph{face detection} step of \emph{video face clustering}. In order to evaluate this approach against the sole use of a traditional CNN, we created a synthetic dataset by projecting images from the FDDB benchmark to equirectangular backgrounds. Our approach, that consists in using viewports with a traditional CNN, performed better than using the traditional CNN directly to the equirectangular images. Moreover, we proposed an authoring model that allows authors to design and create
interactive 360 videos. The proposed model is the result of the analysis of different scenarios of immersive 360 multimedia applications. Besides supporting subtitles positioning in 360-videos, it also supports navigation among 360-videos, additional media, etc. For other case studies not closely related to this dissertation, please check \cite{mendes2020authoring}. Currently, the model can be easily used through text editors with an implementation integrated with the Unity Game Engine\footnote{\url{https://github.com/TeleMidia/VR360Authoring}}.

The choice for these three applications was due to observations of opportunities to apply \emph{Video Face Clustering} in different contexts. We started with Video Face Recognition. When we observed that the the use of only the \emph{Video Face Clustering} part of the method could generate the time segments that each actor is present without having to identify them, we hypothesised that this particular feature could also be used to clustering actors in different videos. From this observation, we had the idea of developing a method for \emph{Educational Video Recommendation} using this clustering of actors in different videos. The idea for \emph{Subtitles Positioning in 360-video} came from the authoring model we developed. We observed that we could automatically identify the actors using \emph{Video Face Clustering} so that the subtitles could follow the speakers in the 360-video. With that in mind, we adapted \emph{Video Face Clustering} for the 360-video setting. 

\section{Publications}

 As results of this research, three papers have already been published at relevant multimedia conferences \cite{mendes2020cluster,mendes2020ISM, mendes2020authoring}. In \cite{mendes2020cluster}, we have evaluated video face clustering together with a cluster-matching method for video face recognition. In \cite{mendes2020ISM}, we have used video face clustering and the presence of actors in different video as a mean for recommending educational videos. In \cite{mendes2020authoring}, we have developed an authoring model and a player for interactive 360-video. 

\arial
\bibliography{references}

% Apendix chapters below.
%\normalfont
%\appendix
%\input{appendix.tex}

\bibliographystyle{bibstyles/IEEEtran}

\end{document}